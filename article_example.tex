\documentclass[a4paper,12pt]{article}

\usepackage[utf8]{inputenc}
\usepackage{graphicx} % pacchetto per parti grafiche
\usepackage{color}
\usepackage{lineno}
\usepackage{hyperref}
\usepackage{natbib}
\usepackage{listings}
\linenumbers
%  COLORI TESTO 
%  items

\title{This is my first LaTeX document!}
\author{Duccio Rocchini $^1$, Student2 $^2$}

% le funzioni in latex si indicano con \ e gli argomenti vanno all'interno di parentesi graffe

\begin{document}
\maketitle

$^1$ Alma Mater Studiorum University of Bologna

$^2$ NASA

\tableofcontents

\begin{abstract}
Here is the abstract of my Master Thesis.

The Master Thesis is dealing with something.

Concluding, I made beautiful science.
\end{abstract}


\section{Introduction}\label{section:intro}
\textcolor{blue}{This is my master thesis blablabla.}

As we will se later on, the formula for calculating diversity is quite simple and makes use of a $\sum$.

Blablabla ......

The flight of laepidoptera is based on their physical characters \citep{parlin}. This is also true for the shape of phytoplankton \citep{jones}.

\section{Methods}
These are the methods of my thesis. As I stated in section \ref{section:intro}....

The diversity was calculated by the following formula:

\begin{equation}
 H = 3 \times 2 
\label{eq:moltiplica}
\end{equation}

\begin{equation}
 H = - \sum_{i=1}^{6} p_i \times \log{p_{ijm}}
\label{eq:shannon}
\end{equation}

And here is the code used!:
\lstinputlisting[language=R]{source_test_lezione.r}

\subsection{Study area}
My study area is the Brenta Chain...

\subsection{Field sampling}
I sampled these plots (Figure \ref{fig:canyon}).

\begin{figure}
\centering
    \includegraphics[width=0.7\textwidth]{climbing.png}
    \caption{This is the Grand Canyon.}
    \label{fig:canyon}
\end{figure}


\section{Results}
Here are the main results of my paper / thesis:

\begin{itemize}
 \item Laepidoptera are different from phytoplankton
 \item blablabla
 \item blablabla
\end{itemize}

\begin{enumerate}
 \item Laepidoptera are different from phytoplankton
 \item blablabla
 \item blablabla
\end{enumerate}


\section{Discussion}
As we have seen in Equation \ref{eq:shannon} blablabla....
But also Equation \ref{eq:moltiplica} is important.

\begin{thebibliography}{999}

\bibitem[Jones et al.(2021)]{jones}
Jones, C., Clayton, S., Ribalet, F., Armbrust, E.V. and Harchaoui, Z. (2021), A Kernel-Based Change Detection Method to Map Shifts in Phytoplankton Communities Measured by Flow Cytometry. Methods in Ecology and Evolution. Accepted Author Manuscript. https://doi.org/10.1111/2041-210X.13647

\bibitem[Parlin et al.(2021)]{parlin}
Parlin, A.F., Stratton, S.M. and Guerra, P.A. (2021), Assaying lepidopteran flight directionality with non-invasive methods that permit repeated use and release after testing. Methods in Ecology and Evolution. Accepted Author Manuscript. https://doi.org/10.1111/2041-210X.13648

\end{thebibliography}


\end{document}
